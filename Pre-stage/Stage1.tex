\documentclass[journal]{IEEEtran}
\usepackage{sharina}

\author{
    \IEEEauthorblockN{Jinlong Liu}\\
    \IEEEauthorblockA{Department of Computer Science, University of Liverpool
    \\J.Liu157@liverpool.ac.uk}
}
\title{Research Proposal of Visual Spatial Reasoning of Large Language Models}
\begin{document}
\maketitle
\begin{abstract}
Large Language Models has become a hot topic in the field of Natural Language Processing. The Visual Spatial Reasoning is a key ability of human intelligence. In this paper, we will review the recent research on the Visual Spatial Reasoning of Large Language Models.
\end{abstract}
\section{Reasearch Topic}
In recent years, Large Language Models (LLMs) have achieved great success in the field of Natural Language Processing (NLP). The LLMs are trained on large-scale text corpus, and can be used to solve many NLP tasks, such as machine translation, question answering, and text generation. The LLMs are usually trained on text corpus, and can be used to solve many NLP tasks, such as machine translation, question answering, and text generation. \\
However, the LLMs are not good at solving the Spatial Reasoning tasks. In generally, Spatial knowledge have these aspects: including (i) mereotopology, (ii) direction and orientation, (iii) size, (iv) distance and (v) shape \cite{cohn2008qualitative}. So Spatial Reasoning is to reasoning more than one of these aspects of Spatial knowledge at same time.

\bibliographystyle{IEEEtran}
\bibliography{ref}

\end{document}