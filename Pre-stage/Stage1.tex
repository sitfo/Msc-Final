\documentclass[journal,10pt]{IEEEtran}
\usepackage{sharina}

\begin{document}
% Inspired by title template from ShareLaTeX Learn; Gubert Farnsworth & John Doe
% Edited by Jon Arnt Kårstad, NTNU IMT
% Edited by Linda Vogel, HTWK
\begin{titlepage}
\vbox{ }
\vbox{ }
\begin{center}
% Upper part of the page

\includegraphics[width=0.40\textwidth]{liverpool.png}\\[1cm]
%\includegraphics[width=0.40\textwidth,left]{Images/HTWK-Fakultaetszusatz_ing_schwarz_de.png}\\[1cm]

\textsc{\LARGE THE UNIVERSITY OF LIVERPOOL}\\[1.5cm]
\textsc{\Large COMP702-MSC Final Project}\\[0.5cm]
\vbox{ }

% Title


\huge \bfseries Visual Spatial Reasoning for Large Language Models\\[0.4cm]


\begin{table}[h]
\large
    \centering
    \begin{tabular}{ll}
        \emph{Author} & Jinlong Liu \\
        \emph{Student ID} & 201678181\\
        \emph{Supervisor} &  Frank Wolter \\
    \end{tabular}
\end{table}

\vfill
% Bottom of the page
{\large \today}
\end{center}
\end{titlepage}
\section{Project Description}

In recent years, Large Language Models (LLMs) have revolutionized the field of Natural Language Processing (NLP) with their remarkable achievements. These models, trained on vast amounts of text data, have proved highly effective in solving numerous NLP tasks, including machine translation, question answering, and text generation. Their ability to comprehend and generate human-like text has been nothing short of impressive.

However, despite their proficiency in various linguistic tasks, LLMs have encountered challenges when it comes to Spatial Reasoning. Unlike tasks centered purely on language, Spatial Reasoning requires understanding and manipulating visual and spatial information. To shed light on this limitation, I aim to conduct an experiment utilizing popular LLMs like GPT-4. The focus will be on testing their capability in Visual Spatial Reasoning.

The experiment will revolve around a widely used task known as Visual Question Answering (VQA), which demands machines to answer questions based on provided images. For instance, given an image depicting different objects, the machine will be tasked with responding to inquiries such as "What is the spatial relationship between object A and object B?". Through these carefully crafted tests, we can gauge the extent of Spatial Reasoning proficiency exhibited by LLMs.

By assessing their performance in VQA and their ability to comprehend and reason about spatial relationships in visual content, we can obtain valuable insights into the Spatial Reasoning capabilities of LLMs. These findings will not only deepen our understanding of the strengths and limitations of these models but also pave the way for further advancements in the realm of NLP, bringing us closer to more comprehensive and versatile language models

\section{Aims and Objectives}
\subsection{Aims}
\begin{enumerate}
    \item To investigate the Spatial Reasoning abilities of Large Language Models (LLMs), specifically focusing on their performance in Visual Question Answering (VQA) tasks.
    \item To understand the limitations and challenges faced by LLMs in comprehending and reasoning about spatial relationships in visual content.
    \item To assess the current state of Spatial Reasoning capabilities in leading LLMs, then compare with elder models and explore their improvement in this domain. Besides, to identify potential avenues for enhancing LLMs' Spatial Reasoning abilities.
\end{enumerate}
\subsection{Objectives}
\begin{enumerate}
    \item Design and develop a comprehensive experimental framework for evaluating LLMs' Spatial Reasoning abilities in VQA tasks.
    \item Curate a diverse dataset of visual stimuli and corresponding questions that require spatial reasoning skills to answer accurately.
    \item Conduct systematic experiments using the prepared dataset and modified LLMs to assess their performance and measure their level of Spatial Reasoning competence.
    \item Analyze the experimental results to identify patterns, trends, and challenges in LLMs' Spatial Reasoning capabilities.
    \item Provide insights into the strengths and limitations of current LLMs for Spatial Reasoning tasks, based on the experimental findings.
    \item Suggest potential avenues for enhancing LLMs' Spatial Reasoning abilities, such as incorporating multimodal information or novel architectural modifications.
    \item Contribute to the existing body of knowledge in the field of NLP by advancing our understanding of LLMs' Spatial Reasoning abilities and their implications for future research and development.
\end{enumerate}

\section{Key Literature and Background Reading}
For LLMs, the Spatial Reasoning should contain the following aspects: (i) mereotopology,
(ii) direction and orientation, (iii) size, (iv) distance and (v) shape \cite{cohn2008qualitative}. LLMs' answer should contain more than one of these aspects to be considered as a correct answer. Based on this goal, Cohn \textit{et al} \cite{cohn2023dialectical} built a general Spatial Reasoning test which contains Basic spatial relations(parthood, rotation, direction, etc), size shape and location(for instance: \emph{If a circle b is larger than a circle c, is it possible to move b so that it is entirely contained by c?}), Affordances and object interaction and Object permanence, and they found that LLMs give different answers to the same question with continuous seccsions which means LLMs are not consistent in Spatial Reasoning. Lin \textit{et al} \cite{lin2023using} also built a test on the Spatial Reasoning of LLMs, but they are more focused on the spatial relations reperesented by the mathematical language. 

\section{Development and Implementation Summary}
\section{User Interface Mockup}
\section{Data Sources}
\section{Testing}
\section{Evaluation}
\section{Ethical Considerations}
\section{Project Plan}
\section{Risks and Contingency Plans}

\bibliographystyle{IEEEtran}
\bibliography{mybib}

\end{document}