\documentclass[journal,10pt]{IEEEtran}
\usepackage{sharina}

\begin{document}
% Inspired by title template from ShareLaTeX Learn; Gubert Farnsworth & John Doe
% Edited by Jon Arnt Kårstad, NTNU IMT
% Edited by Linda Vogel, HTWK
\begin{titlepage}
\vbox{ }
\vbox{ }
\begin{center}
% Upper part of the page

\includegraphics[width=0.40\textwidth]{liverpool.png}\\[1cm]
%\includegraphics[width=0.40\textwidth,left]{Images/HTWK-Fakultaetszusatz_ing_schwarz_de.png}\\[1cm]

\textsc{\LARGE THE UNIVERSITY OF LIVERPOOL}\\[1.5cm]
\textsc{\Large COMP702-MSC Final Project}\\[0.5cm]
\vbox{ }

% Title


\huge \bfseries Visual Spatial Reasoning for Large Language Models\\[0.4cm]


\begin{table}[h]
\large
    \centering
    \begin{tabular}{ll}
        \emph{Author} & Jinlong Liu \\
        \emph{Student ID} & 201678181\\
        \emph{Supervisor} &  Frank Wolter \\
    \end{tabular}
\end{table}

\vfill
% Bottom of the page
{\large \today}
\end{center}
\end{titlepage}
\section{Project Description}
In recent years, Large Language Models (LLMs) have achieved great success in the field of Natural Language Processing (NLP). The LLMs are trained on large-scale text corpus, and can be used to solve many NLP tasks, such as machine translation, question answering, and text generation. The LLMs are usually trained on text corpus, and can be used to solve many NLP tasks, such as machine translation, question answering, and text generation.\\
However, the LLMs are not good at solving the Spatial Reasoning tasks. In this project, I will design an experiment for most popular LLMs, such as GPT-4, to test their ability of Visual Spatial Reasoning. The experiment will be based on the Visual Question Answering (VQA)--a task that requires the machine to answer questions about the given image. For example, given an image with different objects, the machine needs to answer the question "What the relevant position of the object A to the object B?". With these tests, we can evaluate what level of Spatial Reasoning ability the LLMs have.\\
\section{Aims and Objectives}

\section{Key Literature and Background Reading}
\cite{borji2023categorical}
\section{Development and Implementation Summary}
\section{User Interface Mockup}
\section{Data Sources}
\section{Testing}
\section{Evaluation}
\section{Ethical Considerations}
\section{Project Plan}
\section{Risks and Contingency Plans}

\bibliographystyle{IEEEtran}
\bibliography{mybib}

\end{document}