\documentclass[journal,10pt]{IEEEtran}
\usepackage{sharina}

\begin{document}
% Inspired by title template from ShareLaTeX Learn; Gubert Farnsworth & John Doe
% Edited by Jon Arnt Kårstad, NTNU IMT
% Edited by Linda Vogel, HTWK
\begin{titlepage}
\vbox{ }
\vbox{ }
\begin{center}
% Upper part of the page

\includegraphics[width=0.40\textwidth]{liverpool.png}\\[1cm]
%\includegraphics[width=0.40\textwidth,left]{Images/HTWK-Fakultaetszusatz_ing_schwarz_de.png}\\[1cm]

\textsc{\LARGE THE UNIVERSITY OF LIVERPOOL}\\[1.5cm]
\textsc{\Large COMP702-MSC Final Project}\\[0.5cm]
\vbox{ }

% Title


\huge \bfseries Visual Spatial Reasoning for Large Language Models\\[0.4cm]


\begin{table}[h]
\large
    \centering
    \begin{tabular}{ll}
        \emph{Author} & Jinlong Liu \\
        \emph{Student ID} & 201678181\\
        \emph{Supervisor} &  Frank Wolter \\
    \end{tabular}
\end{table}

\vfill
% Bottom of the page
{\large \today}
\end{center}
\end{titlepage}
\section{Project Description}

In recent years, Large Language Models (LLMs) have revolutionized the field of Natural Language Processing (NLP) with their remarkable achievements. These models, trained on vast amounts of text data, have proved highly effective in solving numerous NLP tasks, including machine translation, question answering, and text generation. Their ability to comprehend and generate human-like text has been nothing short of impressive.

However, despite their proficiency in various linguistic tasks, LLMs have encountered challenges when it comes to Spatial Reasoning. Unlike tasks centered purely on language, Spatial Reasoning requires understanding and manipulating visual and spatial information. To shed light on this limitation, I aim to conduct an experiment utilizing popular LLMs like GPT-4. The focus will be on testing their capability in Visual Spatial Reasoning.

The experiment will revolve around a widely used task known as Visual Question Answering (VQA), which demands machines to answer questions based on provided images. For instance, given an image depicting different objects, the machine will be tasked with responding to inquiries such as "What is the spatial relationship between object A and object B?". Through these carefully crafted tests, we can gauge the extent of Spatial Reasoning proficiency exhibited by LLMs.

By assessing their performance in VQA and their ability to comprehend and reason about spatial relationships in visual content, we can obtain valuable insights into the Spatial Reasoning capabilities of LLMs. These findings will not only deepen our understanding of the strengths and limitations of these models but also pave the way for further advancements in the realm of NLP, bringing us closer to more comprehensive and versatile language models

\section{Aims and Objectives}
\subsection{Aims}
\begin{enumerate}
    \item To investigate the Spatial Reasoning abilities of Large Language Models (LLMs), specifically focusing on their performance in Visual Question Answering (VQA) tasks.
    \item To understand the limitations and challenges faced by LLMs in comprehending and reasoning about spatial relationships in visual content.
    \item To assess the current state of Spatial Reasoning capabilities in leading LLMs, then compare with elder models and explore their improvement in this domain. Besides, to identify potential avenues for enhancing LLMs' Spatial Reasoning abilities.
\end{enumerate}
\subsection{Objectives}
\begin{enumerate}
    \item Design and develop a comprehensive experimental framework for evaluating LLMs' Spatial Reasoning abilities in VQA tasks.
    \item Curate a diverse dataset of visual stimuli and corresponding questions that require spatial reasoning skills to answer accurately.
    \item Conduct systematic experiments using the prepared dataset and modified LLMs to assess their performance and measure their level of Spatial Reasoning competence.
    \item Analyze the experimental results to identify patterns, trends, and challenges in LLMs' Spatial Reasoning capabilities.
    \item Provide insights into the strengths and limitations of current LLMs for Spatial Reasoning tasks, based on the experimental findings.
    \item Suggest potential avenues for enhancing LLMs' Spatial Reasoning abilities, such as incorporating multimodal information or novel architectural modifications.
    \item Contribute to the existing body of knowledge in the field of NLP by advancing our understanding of LLMs' Spatial Reasoning abilities and their implications for future research and development.
\end{enumerate}

\section{Key Literature and Background Reading}
For Large Language Models (LLMs), Spatial Reasoning encompasses several key aspects, including (i) mereotopology, (ii) direction and orientation, (iii) size, (iv) distance, and (v) shape, as discussed by Cohn et al. (2008) in their qualitative analysis \cite{cohn2008qualitative}. To assess LLMs' performance in Spatial Reasoning, Cohn et al. (2023) developed a comprehensive Spatial Reasoning test that incorporates basic spatial relations (such as parthood, rotation, and direction), size, shape, location, affordances, object interaction, and object permanence \cite{cohn2023dialectical}. Their findings revealed inconsistencies in LLMs' responses to the same question during continuous sessions, indicating that LLMs struggle with maintaining consistency in Spatial Reasoning tasks.

In a related study by Lin et al. (2023), a test was devised to evaluate LLMs' Spatial Reasoning abilities with a focus on mathematical representation of questions \cite{lin2023using}. Their research demonstrated that while LLMs may not excel at directly answering spatial questions, they exhibit proficiency in transforming questions into mathematical forms based on user demands.

Moreover, Borji (2023) proposed a failure task category for ChatGPT, within which Spatial Reasoning was highlighted \cite{borji2023categorical}. In this task, the researcher provided a grid description and asked the model to find a path from a start point to an end point. However, the model failed to answer the question accurately, highlighting its limitations in Spatial Reasoning tasks.

Collectively, these studies shed light on the challenges LLMs face when it comes to Spatial Reasoning, including inconsistencies in responses, struggles with direct spatial question answering, and difficulties in solving specific spatial tasks like path finding. 

There is another direction Visual Spatial Reasoning(VSR) is quite related to Spatial Reasoning of LLMs, but VSR is mostly on Visual Language Models(VLMs), which is a different topic. However, it is still worth mentioning some of the related works. In this direction, researcher focus on the ability of VLMs to understand the commonsense spatial reasoning of VLMs, so there are lots of benchmark or question set for VLMs to test their ability. For example, Liu et al (2023) \cite{liu2022visual} built a dataset with 66 type of spatial relations for more than 10K text-image pairs. And they have tested the accuracy of human being as baseline.

\section{Development and Implementation Summary}
\section{User Interface Mockup}
\section{Data Sources}
\section{Testing}
\section{Evaluation}
\section{Ethical Considerations}
\section{Project Plan}
\section{Risks and Contingency Plans}

\bibliographystyle{IEEEtran}
\bibliography{mybib}

\end{document}